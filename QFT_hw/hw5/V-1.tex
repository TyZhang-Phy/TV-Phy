\documentclass[12pt]{article}
\usepackage{ctex}
\usepackage{graphicx}
\usepackage{subfigure}
\usepackage{caption}
\usepackage{float}
\usepackage{physics}
\usepackage{amsmath}
\usepackage{geometry}
\geometry{left=2.5cm,right=2.5cm,top=2.5cm,bottom=2.5cm}
\title{Solution to Problem V-1}
\author{Zhang Tingyu $\ $ 35206402}
\graphicspath{{figures/}}

\begin{document}

\maketitle

\section*{(a)}

For Dirac Fermion with Lagrangian density
\begin{equation}
    \mathcal{L}=\bar{\Psi}(i\gamma^\mu\partial_\mu-m)\Psi,
\end{equation}
the canonical momentums are derived as 
\begin{equation}
    \begin{split}
        \Pi&=\frac{\partial\mathcal{L}}{\partial\dot{\Psi}}=-i\bar{\Psi}\gamma^0\\
        \bar{\Pi}&=\frac{\partial\mathcal{L}}{\partial\dot{\bar{\Psi}}}=0.
    \end{split}
\end{equation} 
Thus the primary constraints are given by
\begin{equation}
    \chi_1=\Pi+i\bar{\Psi}\gamma^0,\qquad\chi_2=\bar{\Pi}.
\end{equation}
Notice that 
\begin{equation}
    [\Psi(\mathbf{x}),\Pi(\mathbf{y})]_P=\big[\bar{\Psi}(\mathbf{x}),
    \bar{\Pi}(\mathbf{y})\big]_P=\delta^3(\mathbf{x}-\mathbf{y}),
\end{equation}
and $[\chi_1(\mathbf{x}),\chi_2(\mathbf{y})]=
i\gamma^0\delta^3(\mathbf{x}-\mathbf{y})\neq0$, which shows that the constraints 
$\chi_1$ and $\chi_2$ are second class constraints.
\begin{equation}
    C_{1x,2y}=-C_{2y,1x}=i\gamma^0\delta^3(\mathbf{x}-\mathbf{y}).
\end{equation}
\begin{equation}
    (C^{-1})^{1x,2y}=-(C^{-1})^{2y,1x}=-i\gamma^0\int\frac{d^3k}{(2\pi)^3}e^{\mathbf{k}
    \cdot(\mathbf{x}-\mathbf{y})}
\end{equation}
Then the Dirac bracket can be derived by Poission bracket 
\begin{equation}
    [A, B]_{D}=[A, B]_{P}-\left[A, \chi_{N}\right]_{P}
    \left(C^{-1}\right)^{N M}\left[\chi_{M}, B\right]_{P}.
\end{equation}
According to this equation, we can write these Dirac bracket:
\begin{equation}
    \begin{split}
        &\big[\Psi(\mathbf{x}),\bar{\Psi}(\mathbf{y})\big]_D=-i\gamma^0\delta^3
        (\mathbf{x}-\mathbf{y})\\
        &[\Psi(\mathbf{x}),\Pi(\mathbf{y})]_D=\delta^3(\mathbf{x}-\mathbf{y})\\
        &\big[\bar{\Psi}(\mathbf{x}),\Pi(\mathbf{y})\big]_D=0\\
        &\big[\bar{\Psi}(\mathbf{x}),\bar{\Pi}(\mathbf{y})\big]_D=0.
    \end{split}
\end{equation}
\section*{(b)}

For photon with Lagrangian density
\begin{equation}
    \mathcal{L}=-\frac{1}{4}F_{\mu\nu}F^{\mu\nu},
\end{equation}
where
\begin{equation}
    F_{\mu \nu}=\partial_{\mu} A_{\nu}-\partial_{\nu} A_{\mu},
\end{equation}
the canonical momentum fields are derived as 
\begin{equation}
    \Pi^\mu=\frac{\delta\mathcal{L}}{\delta(\partial_0A_\mu)}=-F^{0\mu}.
\end{equation}
The primary constraint is given by
\begin{equation}
    \chi_1=\Pi^0.
\end{equation}
The motion equation gives the secondary constriant 
\begin{equation}
    \partial_\mu\bigg(\frac{\delta\mathcal{L}}{\delta(\partial_0A_\mu)}\bigg)=
    \partial_\mu\Pi^\mu=\partial_i\Pi^i=0
\end{equation}
\begin{equation}
    \Rightarrow \chi_2=\partial_i\Pi^i.
\end{equation}
Obviously, $[\chi_1,\chi_2]_P=0$, which means they are the first class constraints. 
We'll have to fix a gauge to eliminate the corresponding degrees of freedom. In 
Coulomb gauge, we take
\begin{equation}
    \chi_3=\partial_iA^i=0.
\end{equation}
Notice that $[\chi_2,\chi_3]\neq0$. In this system, the current $J_0=0$, thus we 
have the secondary constraint in Coulomb gauge 
\begin{equation}
    \chi_4=\partial^i\partial_iA_0=0.
\end{equation}
We can see that $[\chi_2,\chi_3]\neq0$, which means $\chi_2$ and $\chi_3$ are 
the second class constraints. Meanwhile, $\chi_1$ and $\chi_4$ are first class 
constraints. We set
\begin{equation}
    \chi'_1=\partial_iA^i,\qquad\chi'_2=\partial_i\Pi^i.
\end{equation}
The elements of $C$ matrix can be written
\begin{equation}
    C_{1x,2y}=-C_{2y,1x}=-\partial^i\partial_i\delta^{3}(\mathbf{x}-\mathbf{y}),
\end{equation}
\begin{equation}
    (C^{-1})_{1x,2y}=-(C^{-1})_{2y, 1x}=-\int \frac{d^{3} k}{(2 \pi)^{3}}
    \frac{e^{i \mathbf{k} \cdot(\mathbf{x}-\mathbf{y})}}
    {\mathbf{y}^{2}}=-\frac{1}{4 \pi|\mathbf{x}-\mathbf{y}|}.
\end{equation}
\begin{equation}
    \begin{split}
        &[A^{i}(\boldsymbol{x}), \chi_{2 \boldsymbol{y}}]_{P}=-\frac{\partial}
        {\partial x^{i}} \delta^{3}(\boldsymbol{x}-\boldsymbol{y})\\
        &[\Pi_{i}(\boldsymbol{x}), \chi_{1 \boldsymbol{y}}]_{P}=
        \frac{\partial}{\partial x^{i}} \delta^{3}(\boldsymbol{x}-\boldsymbol{y}).
    \end{split}
\end{equation}
According to Eq.(7), we can write these Dirac brackets
\begin{equation}
    \begin{split}
        &\big[A^\mu(\mathbf{x}),\Pi_\nu(\mathbf{y})\big]_D=(\delta^\mu_{\ \nu}
        -g^{\ 0}_\nu\delta^\mu_{\ 0})\delta^3(\mathbf{x}-\mathbf{y})-
        \partial^\mu\partial_\nu\frac{1}{4\pi|\mathbf{x}-\mathbf{y}|}\\
        &\big[A^\mu(\mathbf{x}),A_\nu(\mathbf{y})\big]_D=0\\
        &\big[\Pi^\mu(\mathbf{x}),\Pi_\nu(\mathbf{y})\big]_D=0.
    \end{split}
\end{equation}

\end{document}