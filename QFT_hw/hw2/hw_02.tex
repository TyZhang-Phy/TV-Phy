\documentclass[12pt]{article}
\usepackage{ctex}
\usepackage{graphicx}
\usepackage{subfigure}
\usepackage{caption}
\usepackage{float}
\usepackage{physics}
\usepackage{amsmath}
\usepackage{geometry}
\geometry{left=2.5cm,right=2.5cm,top=2.5cm,bottom=2.5cm}
\title{Homework 02}
\author{Zhang Tingyu $\ $ 35206402}
\graphicspath{{figures/}}

\begin{document}

\maketitle

\section*{II-2}

\subsection*{(a)}

The Lagrangian of the system is written
\begin{equation}
    \mathcal{L}=(\nabla\Phi)^*(\nabla\Phi)-V(|\Phi|^2)
\end{equation}
which gives the Euler-Lagrange equation:
\begin{equation}
    \nabla\frac{\delta\mathcal{L}}{\delta(\nabla\Phi)}-\frac{\delta
    \mathcal{L}}{\delta\Phi}=0,\ \ \ \ 
    \nabla\frac{\delta\mathcal{L}}{\delta(\nabla\Phi^*)}-\frac{\delta
    \mathcal{L}}{\delta\Phi^*}=0.
\end{equation}
The system is invariant under the phase transformation
\begin{equation*}
    \Phi\rightarrow e^{-i\alpha}\Phi,\ \ \ \ 
    \Phi^*\rightarrow e^{i\alpha}\Phi^*
\end{equation*}
Thus
\begin{equation}
    \delta\mathcal{L}=\frac{\delta\mathcal{L}}{\delta\Phi}\delta\Phi+
    \frac{\delta\mathcal{L}}{\delta(\nabla\Phi)}\delta(\nabla\Phi)+
    \frac{\delta\mathcal{L}}{\delta\Phi^*}\delta\Phi^*+
    \frac{\delta\mathcal{L}}{\delta(\nabla\Phi^*)}\delta(\nabla\Phi^*)=0,
\end{equation}
and according to the Eq.(2), 
\begin{equation}
    \begin{split}
        \delta\mathcal{L}&=\nabla\frac{\delta\mathcal{L}}{\delta(\nabla\Phi)}
        \delta\Phi+\frac{\delta\mathcal{L}}{\delta(\nabla\Phi)}\delta(\nabla
        \Phi)+\nabla\frac{\delta\mathcal{L}}{\delta(\nabla\Phi^*)}\delta\Phi^*+
        \frac{\delta\mathcal{L}}{\delta(\nabla\Phi^*)}\delta(\nabla\Phi^*)\\
        &=-i\nabla(\frac{\delta\mathcal{L}}{\delta(\nabla\Phi)}\Phi)\delta\alpha
        +i\nabla(\frac{\delta\mathcal{L}}{\delta(\nabla\Phi^*)}\Phi^*)\delta
        \alpha=0
    \end{split}
\end{equation}
The Noether current is defined as
\begin{equation}
    \begin{split}
        J_\mu&=-i\frac{\delta\mathcal{L}}{\delta(\partial_\mu\Phi)}\Phi+
        i\frac{\delta\mathcal{L}}{\delta(\partial_\mu\Phi^*)}\Phi^*\\
        &=-i(\partial_\mu\Phi^*)\Phi+i\Phi^*(\partial_\mu\Phi)\\
        &=-i[(\partial_\mu\Phi^*)\Phi-\Phi^*(\partial_\mu\Phi)]\\
        &=2{\rm Im}[(\partial_\mu\Phi^*)\Phi].
    \end{split}
\end{equation}
And the Noether charge is 
\begin{equation}
    \begin{split}
        N&=\int d^3xJ_0=2\int d^3x{\rm Im}[(\partial_t\Phi^*)\Phi]\\
        &=2\int d^3x{\rm Im}[\pi^*\Phi]
    \end{split}
\end{equation}

We require the complex scalar field $\Phi$ and its conjugate $\pi$ to be 
\begin{equation*}
    \Phi=\Phi_1+i\Phi_2,\ \ \ \ \pi=\pi_1+i\pi_2
\end{equation*}
The Hamiltonian can be written as 
\begin{equation}
    \begin{split}
        H&=\int d^dx\big[\pi^*\pi+(\nabla\Phi)^*(\nabla\Phi)+V(|\Phi|^2)\big]\\
        &=\int d^dx\big[\pi_1^2+\pi_2^2+(\nabla\Phi_1)^2+(\nabla\Phi_2)^2+
        V(|\Phi|^2)\big].
    \end{split}
\end{equation}

And the chemical potential is 

\begin{equation}
    \begin{split}
        -\mu N&=-2\mu\int d^3x {\rm Im}[\pi^*\Phi]\\
        &=-2\mu\int d^3x {\rm Im}[(\pi_1-i\pi_2)(\Phi_1+i\Phi_2)]\\
        &=-2\mu\int d^3x(\pi_1\Phi_2-\pi_2\Phi_1)
    \end{split}
\end{equation}

Then we can write the partition function in path integral form
\begin{equation}
    \begin{split}
        Z=&\int \mathcal{D}\pi\mathcal{D}\pi^*\mathcal{D}\Phi\mathcal{D}\Phi^*
        \exp\bigg[-\int_0^\beta d\tau(H-\mu N)+i\int d^{d+1}x((\pi(\partial_\tau
        \Phi^*)+\pi^*(\partial_\tau\Phi)))\bigg]\\
        =&\int \mathcal{D}\pi\mathcal{D}\pi^*\mathcal{D}\Phi\mathcal{D}\Phi^*
        \exp\bigg\{-\int d^{d+1}x\Big[\pi_1^2+\pi_2^2+(\nabla\Phi_1)^2+(\nabla
        \Phi_2)^2+V(|\Phi|^2)\\
        &-2\mu(\pi_1\Phi_2-\pi_2\Phi_1)-2i\pi_1\partial_\tau
        \Phi_1-2i\pi_2\partial_\tau\Phi_2\Big]\bigg\}\\
        =&\int \mathcal{D}\pi\mathcal{D}\pi^*\mathcal{D}\Phi\mathcal{D}\Phi^*
        \exp\bigg\{-\int d^{d+1}x\Big[\pi_1^2+2\pi_1(-i\partial_\tau\Phi_1
        -\mu\Phi_2)+\pi_2^2\\
        &+2\pi_2(-i\partial_\tau\Phi_2+\mu\Phi_1)+(\nabla\Phi_1)^2
        +(\nabla\Phi_2)^2+V(|\Phi|^2)\Big]\bigg\}.
    \end{split}
\end{equation}

Now carry out the Gaussian integral over $\pi_1$ and $\pi_2$ in Eq.(9),

\begin{equation}
    \begin{split}
        &\int d\pi_1\exp\bigg\{-\int d^{d+1}x\Big[\pi_1^2+2\pi_1
        (-i\partial_\tau\Phi_1-\mu\Phi_2)\Big]\bigg\}\\
        =&C\exp[-\int d^{d+1}x(\partial_\tau\Phi_1-i\mu\Phi_2)^2]
    \end{split}
\end{equation}
\begin{equation}
    \begin{split}
        &\int d\pi_2\exp\bigg\{-\int d^{d+1}x\Big[\pi_2^2+2\pi_2
        (-i\partial_\tau\Phi_2+\mu\Phi_1)\Big]\bigg\}\\
        =&C\exp[-\int d^{d+1}x(\partial_\tau\Phi_2+i\mu\Phi_1)^2]
    \end{split}
\end{equation}
And the partition function becomes
\begin{equation}
    \begin{split}
        Z&=C\int \mathcal{D}\Phi\mathcal{D}\Phi^*\exp\bigg\{-\int d^{d+1}x
        \Big[(\partial_\tau\Phi_1-i\mu\Phi_2)^2+(\partial_\tau\Phi_2
        +i\mu\Phi_1)^2\\
        &+(\nabla\Phi_1)^2+(\nabla\Phi_2)^2+V(|\Phi|^2)\Big]\bigg\}\\
        &=C\int \mathcal{D}\Phi\mathcal{D}\Phi^*\exp\bigg\{-\int d^{d+1}x
        \Big[(\partial_\tau+\mu)\Phi^*(\partial_\tau-\mu)\Phi+(\nabla\Phi)^*
        (\nabla\Phi)+V(|\Phi|^2)\Big]\bigg\}
    \end{split}
\end{equation}

\subsection*{(b)}

With the field redefinition
\begin{equation}
    \Phi'(x,t)=e^{i\mu t}\Phi
\end{equation}
the time derivation of the new field is given by
\begin{equation}
    \begin{split}
        &\partial_\tau\Phi'=\mu\Phi'+e^{i\mu t}\partial_\tau\Phi\\
        &\partial_\tau\Phi'^*=-\mu\Phi'^*+e^{-i\mu t}\partial_\tau\Phi^*
    \end{split}
\end{equation}
and the modified partition function becomes
\begin{equation}
    \begin{split}
        Z&=C\int \mathcal{D}\Phi'\mathcal{D}\Phi'^*\exp\bigg\{-\int d^{d+1}x
        \Big[(\partial_\tau+\mu)\Phi'^*(\partial_\tau-\mu)\Phi'+(\nabla\Phi')^*
        (\nabla\Phi')\\
        &+V(|\Phi'|^2)\Big]\bigg\}\\
        &=C\int \mathcal{D}\Phi\mathcal{D}\Phi^*\exp\bigg\{-\int d^{d+1}x
        \Big[\partial_\tau\Phi^*\partial_\tau\Phi+(\nabla\Phi)^*
        (\nabla\Phi)+V(|\Phi|^2)\Big]\bigg\},
    \end{split}
\end{equation}
which is equivalent to the path integral without the modification.

\end{document}